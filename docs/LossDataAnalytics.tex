\documentclass[]{article}
\usepackage{lmodern}
\usepackage{amssymb,amsmath}
\usepackage{ifxetex,ifluatex}
\usepackage{fixltx2e} % provides \textsubscript
\ifnum 0\ifxetex 1\fi\ifluatex 1\fi=0 % if pdftex
  \usepackage[T1]{fontenc}
  \usepackage[utf8]{inputenc}
\else % if luatex or xelatex
  \ifxetex
    \usepackage{mathspec}
  \else
    \usepackage{fontspec}
  \fi
  \defaultfontfeatures{Ligatures=TeX,Scale=MatchLowercase}
\fi
% use upquote if available, for straight quotes in verbatim environments
\IfFileExists{upquote.sty}{\usepackage{upquote}}{}
% use microtype if available
\IfFileExists{microtype.sty}{%
\usepackage{microtype}
\UseMicrotypeSet[protrusion]{basicmath} % disable protrusion for tt fonts
}{}
\usepackage[margin=1in]{geometry}
\usepackage{hyperref}
\hypersetup{unicode=true,
            pdfborder={0 0 0},
            breaklinks=true}
\urlstyle{same}  % don't use monospace font for urls
\usepackage{natbib}
\bibliographystyle{plainnat}
\usepackage{longtable,booktabs}
\usepackage{graphicx,grffile}
\makeatletter
\def\maxwidth{\ifdim\Gin@nat@width>\linewidth\linewidth\else\Gin@nat@width\fi}
\def\maxheight{\ifdim\Gin@nat@height>\textheight\textheight\else\Gin@nat@height\fi}
\makeatother
% Scale images if necessary, so that they will not overflow the page
% margins by default, and it is still possible to overwrite the defaults
% using explicit options in \includegraphics[width, height, ...]{}
\setkeys{Gin}{width=\maxwidth,height=\maxheight,keepaspectratio}
\IfFileExists{parskip.sty}{%
\usepackage{parskip}
}{% else
\setlength{\parindent}{0pt}
\setlength{\parskip}{6pt plus 2pt minus 1pt}
}
\setlength{\emergencystretch}{3em}  % prevent overfull lines
\providecommand{\tightlist}{%
  \setlength{\itemsep}{0pt}\setlength{\parskip}{0pt}}
\setcounter{secnumdepth}{5}
% Redefines (sub)paragraphs to behave more like sections
\ifx\paragraph\undefined\else
\let\oldparagraph\paragraph
\renewcommand{\paragraph}[1]{\oldparagraph{#1}\mbox{}}
\fi
\ifx\subparagraph\undefined\else
\let\oldsubparagraph\subparagraph
\renewcommand{\subparagraph}[1]{\oldsubparagraph{#1}\mbox{}}
\fi

%%% Use protect on footnotes to avoid problems with footnotes in titles
\let\rmarkdownfootnote\footnote%
\def\footnote{\protect\rmarkdownfootnote}

%%% Change title format to be more compact
\usepackage{titling}

% Create subtitle command for use in maketitle
\providecommand{\subtitle}[1]{
  \posttitle{
    \begin{center}\large#1\end{center}
    }
}

\setlength{\droptitle}{-2em}

  \title{}
    \pretitle{\vspace{\droptitle}}
  \posttitle{}
    \author{}
    \preauthor{}\postauthor{}
    \date{}
    \predate{}\postdate{}
  
\usepackage{booktabs}
\setcounter{secnumdepth}{2}

\begin{document}

{
\setcounter{tocdepth}{3}
\tableofcontents
}
\section{Introduction to Interactive
Features}\label{introduction-to-interactive-features}

\subsection{Active Learning}\label{active-learning}

\href{https://cei.umn.edu/support-services/tutorials/what-active-learning}{Active
Learning} is a phrase used by educators to mean an approach to classroom
activities in which students engage the material they study through
reading, writing, talking, listening, and reflecting. In contrast, a
traditional instructor is sometimes referred to as the ``sage on the
stage,'' where the teacher does most of the talking and students are
passive.

Proponents of active learning suggest augmenting classroom activities by
talking and listening in small groups, and having students write, read,
and reflect on material. A variety of classroom tools have been
forwarded for engaging students including discussion of scenarios and
case studies, one minute papers, ``shared brainstorming,'' and the like.
For more details, see a site sponsored by the University of Minnesota
that provides \href{https://cei.umn.edu/active-learning}{a list of basic
active learning activities} or a
\href{https://cft.vanderbilt.edu/active-learning/}{summary} from the
University of Vanderbilt.

Active learning is typically used in reference to classroom activities
but it can also refer to textbooks. In our implementation of \emph{Loss
Data Analytics}, we propose to include a number of \textbf{interactive}
features that will allow a student to actively explore the content,
described in Section \ref{S:Features}.

When taken outside of the classroom, the concept of active learning
promotes one of the
\href{http://www.hewlett.org/programs/education/deeper-learning}{deeper
learning} goals set forth by some educational leaders: The ability to
learn how to learn independently. We hope that instructors will be able
to use the online text so that students can discover how to monitor and
direct their own work and learning.

\subsection{LDA Interactive Features}\label{S:Features}

On the one hand, interactive features will help keep readers engaged as
they explore the text. On the other hand, more is not necessarily
better. Readers can be easily overwhelmed with a plethora of options and
not appreciated the pedagogic approach with too many alternatives. So,
we need to be thoughtful as we introduce interactive features.

\subsubsection{What we Have}\label{what-we-have}

One easy feature that we have already adopted extensively is to layer
the content through the use of \emph{Show/Hide} labels or text. Clicking
on text that is bold, in different fonts, as well as color, allows
viewers to reveal and to hide selected material. We use:

\begin{itemize}
\tightlist
\item
  Statistical (\texttt{R}) code available via \emph{Show/Hide} labels
\item
  Exercise Solutions available via \emph{Show/Hide} labels
\item
  Short demonstrations (``snippets'') of mathematics available via
  \emph{Show/Hide} labels
\end{itemize}

Another feature (not so easy to code) that we have introduced is quizzes
that appear at the end of each section. These quizzes are low-level
formative assessment tools; they allow viewers to check their
comprehension of material that they have just read. More details on
\textbf{End of Section Quizzes} appear in Section
\ref{S:EndSectionQuizzes} of this summary.

We have also incorporated a number of \textbf{Animated Graphs} graphs in
the simulation Chapter 6.
\href{https://openacttexts.github.io/Loss-Data-Analytics/C-Simulation.html\#S:ImportanceSampling}{Here}
is one demonstration. These are coded in \texttt{R} - although moving
(animated), they do not require user interaction.

\subsubsection{In Process}\label{in-process}

We are currently in the process of developing an \textbf{Online
Glossary}. This is a feature that viewers regularly see in websites and
so will not be overwhelming. The idea is to move the cursor/mouse over
selected terms/phrases and have a definition appear. This is handy for
insurance and statistical terms, as well as reminders of selected
acronyms.

One of our big challenges will be to decide upon the type and extent of
\textbf{Exercises} to include. See the Section \ref{S:Exercises} for a
summary.

\subsubsection{Plans for Future Work}\label{plans-for-future-work}

There are a number of features that we now know how to implement in our
environment (using \texttt{R} \textbf{Bookdown} package with Github) but
have not yet initiated. In some cases, it is simply a matter of
recruiting volunteers to author the content. In other cases, we need to
convince ourselves that these additions will actually improve the work.

\begin{itemize}
\item
  \textbf{Video explanation of concepts}. This is easy to do with our
  system - see
  \href{https://ewfreesres.github.io/RegressModel/index.html}{the site}
  for an example. The challenge is to recruit people willing to author
  the content.
\item
  \textbf{Interactive statistical code}. As we move from a world of
  mathematical/probabilistic modeling to one that features data, we need
  to think more about how to integrate statistcal code into the
  pedagogy. See Section \ref{S:StatisticalCode} for a summary.
\item
  \textbf{Interactive graphs}. The \texttt{R} package \textbf{Shiny}
  allows us to introduce interactive graphs. See
  \href{https://ewfrees.github.io/LDARcode/index.html\#32_gamma_distribution}{the
  site} for an example. Part of the issue is getting volunteers to
  author the content. Another part is that this package requires a bit
  more overhead so it is not clear as to whether this gets integrated
  into the book or is placed in a supporting site.
\item
  \textbf{Links to external sources}. This is easy to do with our
  system. The challenge is getting the right set of references and
  setting up a system so that links can be maintained.
\end{itemize}

\subsection{LDA Non-Interactive Features}\label{S:NonFeatures}

Simply to round out the discussion, we point out features that are not
interactive that still help to set the book project apart as special.
These include:

\begin{itemize}
\tightlist
\item
  Collobarative, international author and review teams. There a variety
  of points of view represented in this work, making it stronger than if
  authored by an individual.
\item
  Offline versions of the text, via \emph{pdf} and \emph{EPUB}, are
  freely available.
\item
  Translations of the work will begin shortly, focusing on a Spanish
  version but with a Portugese one also in the works.
\end{itemize}

\section{End of Section Quizzes}\label{S:EndSectionQuizzes}

\subsection{1. Introduction to Loss Data
Analytis}\label{introduction-to-loss-data-analytis}

\subsubsection{Section 1.1}\label{section-1.1}

\hypertarget{surveyElement11}{}

\hypertarget{surveyResult11}{}

Show Quiz Solution

\hypertarget{display.Quiz11.2}{}
\begin{center}\rule{0.5\linewidth}{\linethickness}\end{center}

\subsubsection{Section 1.2}\label{section-1.2}

\hypertarget{surveyElement12}{}

\hypertarget{surveyResult12}{}

Show Quiz Solution

\hypertarget{display.Quiz12.2}{}
\begin{center}\rule{0.5\linewidth}{\linethickness}\end{center}

\subsubsection{Section 1.3}\label{section-1.3}

\hypertarget{surveyElement13}{}

\hypertarget{surveyResult13}{}

Show Quiz Solution

\hypertarget{display.Quiz13.2}{}
\begin{center}\rule{0.5\linewidth}{\linethickness}\end{center}

\subsection{2. Frequency Modeling}\label{frequency-modeling}

\subsubsection{Section 2.1}\label{section-2.1}

\hypertarget{surveyElement21}{}

\hypertarget{surveyResult21}{}

Show Quiz Solution

\hypertarget{display.Quiz21.2}{}
\begin{center}\rule{0.5\linewidth}{\linethickness}\end{center}

\subsubsection{Section 2.2}\label{section-2.2}

\hypertarget{surveyElement22}{}

\hypertarget{surveyResult22}{}

Show Quiz Solution

\hypertarget{display.Quiz22.2}{}
\begin{center}\rule{0.5\linewidth}{\linethickness}\end{center}

\subsubsection{Section 2.3}\label{section-2.3}

\hypertarget{surveyElement23}{}

\hypertarget{surveyResult23}{}

Show Quiz Solution

\hypertarget{display.Quiz23.2}{}
\begin{center}\rule{0.5\linewidth}{\linethickness}\end{center}

\subsubsection{Section 2.4}\label{section-2.4}

\hypertarget{surveyElement24}{}

\hypertarget{surveyResult24}{}

Show Quiz Solution

\hypertarget{display.Quiz24.2}{}
\begin{center}\rule{0.5\linewidth}{\linethickness}\end{center}

\subsubsection{Section 2.5}\label{section-2.5}

\hypertarget{surveyElement25}{}

\hypertarget{surveyResult25}{}

Show Quiz Solution

\hypertarget{display.Quiz25.2}{}
\begin{center}\rule{0.5\linewidth}{\linethickness}\end{center}

\subsubsection{Section 2.6}\label{section-2.6}

\hypertarget{surveyElement26}{}

\hypertarget{surveyResult26}{}

Show Quiz Solution

\hypertarget{display.Quiz26.2}{}
\begin{center}\rule{0.5\linewidth}{\linethickness}\end{center}

\subsubsection{Section 2.7}\label{section-2.7}

\hypertarget{surveyElement27}{}

\hypertarget{surveyResult27}{}

Show Quiz Solution

\hypertarget{display.Quiz27.2}{}
\begin{center}\rule{0.5\linewidth}{\linethickness}\end{center}

\subsection{3. Modeling Loss Severity}\label{modeling-loss-severity}

\subsubsection{Section 3.1}\label{section-3.1}

\hypertarget{surveyElement31}{}

\hypertarget{surveyResult31}{}

Show Quiz Solution

\hypertarget{display.Quiz31.2}{}
\begin{center}\rule{0.5\linewidth}{\linethickness}\end{center}

\subsubsection{Section 3.2}\label{section-3.2}

\hypertarget{surveyElement32}{}

\hypertarget{surveyResult32}{}

Show Quiz Solution

\hypertarget{display.Quiz32.2}{}
\begin{center}\rule{0.5\linewidth}{\linethickness}\end{center}

\subsubsection{Section 3.3}\label{section-3.3}

\hypertarget{surveyElement33}{}

\hypertarget{surveyResult33}{}

Show Quiz Solution

\hypertarget{display.Quiz33.2}{}
\begin{center}\rule{0.5\linewidth}{\linethickness}\end{center}

\subsubsection{Section 3.4}\label{section-3.4}

\hypertarget{surveyElement34}{}

\hypertarget{surveyResult34}{}

Show Quiz Solution

\hypertarget{display.Quiz34.2}{}
\begin{center}\rule{0.5\linewidth}{\linethickness}\end{center}

\subsubsection{Section 3.5}\label{section-3.5}

\hypertarget{surveyElement35}{}

\hypertarget{surveyResult35}{}

Show Quiz Solution

\hypertarget{display.Quiz35.2}{}
\begin{center}\rule{0.5\linewidth}{\linethickness}\end{center}

\subsection{4. Model Selection and
Estimation}\label{model-selection-and-estimation}

\subsubsection{Section 4.1}\label{section-4.1}

\hypertarget{surveyElement41}{}

\hypertarget{surveyResult41}{}

Show Quiz Solution

\hypertarget{display.Quiz41.2}{}
\begin{center}\rule{0.5\linewidth}{\linethickness}\end{center}

\subsubsection{Section 4.2}\label{section-4.2}

\hypertarget{surveyElement42}{}

\hypertarget{surveyResult42}{}

Show Quiz Solution

\hypertarget{display.Quiz42.2}{}
\begin{center}\rule{0.5\linewidth}{\linethickness}\end{center}

\subsubsection{Section 4.3}\label{section-4.3}

\hypertarget{surveyElement43}{}

\hypertarget{surveyResult43}{}

Show Quiz Solution

\hypertarget{display.Quiz43.2}{}
\begin{center}\rule{0.5\linewidth}{\linethickness}\end{center}

\subsubsection{Section 4.4}\label{section-4.4}

\hypertarget{surveyElement44}{}

\hypertarget{surveyResult44}{}

Show Quiz Solution

\hypertarget{display.Quiz44.2}{}
\begin{center}\rule{0.5\linewidth}{\linethickness}\end{center}

\subsection{5. Aggregate Loss Models}\label{aggregate-loss-models}

\subsubsection{Section 5.1}\label{section-5.1}

\hypertarget{surveyElement51}{}

\hypertarget{surveyResult51}{}

Show Quiz Solution

\hypertarget{display.Quiz51.2}{}
\begin{center}\rule{0.5\linewidth}{\linethickness}\end{center}

\subsubsection{Section 5.2}\label{section-5.2}

\hypertarget{surveyElement52}{}

\hypertarget{surveyResult52}{}

Show Quiz Solution

\hypertarget{display.Quiz52.2}{}
\begin{center}\rule{0.5\linewidth}{\linethickness}\end{center}

\subsubsection{Section 5.3}\label{section-5.3}

\hypertarget{surveyElement53}{}

\hypertarget{surveyResult53}{}

Show Quiz Solution

\hypertarget{display.Quiz53.2}{}
\begin{center}\rule{0.5\linewidth}{\linethickness}\end{center}

\subsubsection{Section 5.4}\label{section-5.4}

\hypertarget{surveyElement54}{}

\hypertarget{surveyResult54}{}

Show Quiz Solution

\hypertarget{display.Quiz54.2}{}
\begin{center}\rule{0.5\linewidth}{\linethickness}\end{center}

\subsubsection{Section 5.5}\label{section-5.5}

\hypertarget{surveyElement55}{}

\hypertarget{surveyResult55}{}

Show Quiz Solution

\hypertarget{display.Quiz55.2}{}
\begin{center}\rule{0.5\linewidth}{\linethickness}\end{center}

\subsection{8. Risk Classification}\label{risk-classification}

\subsubsection{Section 8.1}\label{section-8.1}

\hypertarget{surveyElement81}{}

\hypertarget{surveyResult81}{}

Show Quiz Solution

\hypertarget{display.Quiz81.2}{}
\begin{center}\rule{0.5\linewidth}{\linethickness}\end{center}

\subsubsection{Section 8.2}\label{section-8.2}

\hypertarget{surveyElement82}{}

\hypertarget{surveyResult82}{}

Show Quiz Solution

\hypertarget{display.Quiz82.2}{}
\begin{center}\rule{0.5\linewidth}{\linethickness}\end{center}

\subsubsection{Section 8.3}\label{section-8.3}

\hypertarget{surveyElement83}{}

\hypertarget{surveyResult83}{}

Show Quiz Solution

\hypertarget{display.Quiz83.2}{}
\begin{center}\rule{0.5\linewidth}{\linethickness}\end{center}

\subsection{9. Experience Rating Using Credibility
Theory}\label{experience-rating-using-credibility-theory}

\subsubsection{Section 9.1}\label{section-9.1}

\hypertarget{surveyElement91}{}

\hypertarget{surveyResult91}{}

Show Quiz Solution

\hypertarget{display.Quiz91.2}{}
\begin{center}\rule{0.5\linewidth}{\linethickness}\end{center}

\subsubsection{Section 9.2}\label{section-9.2}

\hypertarget{surveyElement92}{}

\hypertarget{surveyResult92}{}

Show Quiz Solution

\hypertarget{display.Quiz92.2}{}
\begin{center}\rule{0.5\linewidth}{\linethickness}\end{center}

\subsubsection{Section 9.3}\label{section-9.3}

\hypertarget{surveyElement93}{}

\hypertarget{surveyResult93}{}

Show Quiz Solution

\hypertarget{display.Quiz93.2}{}
\begin{center}\rule{0.5\linewidth}{\linethickness}\end{center}

\subsubsection{Section 9.6}\label{section-9.6}

\hypertarget{surveyElement96}{}

\hypertarget{surveyResult96}{}

Show Quiz Solution

\hypertarget{display.Quiz96.2}{}
\begin{center}\rule{0.5\linewidth}{\linethickness}\end{center}

\subsection{10. Insurance Portfolio Management including
Reinsurance}\label{insurance-portfolio-management-including-reinsurance}

\subsubsection{Section 10.1}\label{section-10.1}

\hypertarget{surveyElement101}{}

\hypertarget{surveyResult101}{}

Show Quiz Solution

\hypertarget{display.Quiz101.2}{}
\begin{center}\rule{0.5\linewidth}{\linethickness}\end{center}

\subsubsection{Section 10.2}\label{section-10.2}

\hypertarget{surveyElement102}{}

\hypertarget{surveyResult102}{}

Show Quiz Solution

\hypertarget{display.Quiz102.2}{}
\begin{center}\rule{0.5\linewidth}{\linethickness}\end{center}

\subsubsection{Section 10.3}\label{section-10.3}

\hypertarget{surveyElement103}{}

\hypertarget{surveyResult103}{}

Show Quiz Solution

\hypertarget{display.Quiz103.2}{}
\begin{center}\rule{0.5\linewidth}{\linethickness}\end{center}

\subsection{13. Data and Systems}\label{data-and-systems}

\subsubsection{Section 13.1}\label{section-13.1}

\hypertarget{surveyElement131}{}

\hypertarget{surveyResult131}{}

Show Quiz Solution

\hypertarget{display.Quiz131.2}{}
\begin{center}\rule{0.5\linewidth}{\linethickness}\end{center}

\subsubsection{Section 13.2}\label{section-13.2}

\hypertarget{surveyElement132}{}

\hypertarget{surveyResult132}{}

Show Quiz Solution

\hypertarget{display.Quiz132.2}{}
\begin{center}\rule{0.5\linewidth}{\linethickness}\end{center}

\subsubsection{Section 13.3}\label{section-13.3}

\hypertarget{surveyElement133}{}

\hypertarget{surveyResult133}{}

Show Quiz Solution

\hypertarget{display.Quiz133.2}{}
\begin{center}\rule{0.5\linewidth}{\linethickness}\end{center}

\subsection{14. Dependence Modeling}\label{dependence-modeling}

\subsubsection{Section 14.1}\label{section-14.1}

\hypertarget{surveyElement141}{}

\hypertarget{surveyResult141}{}

Show Quiz Solution

\hypertarget{display.Quiz141.2}{}
\begin{center}\rule{0.5\linewidth}{\linethickness}\end{center}

\subsubsection{Section 14.2}\label{section-14.2}

\hypertarget{surveyElement142}{}

\hypertarget{surveyResult142}{}

Show Quiz Solution

\hypertarget{display.Quiz142.2}{}
\begin{center}\rule{0.5\linewidth}{\linethickness}\end{center}

\subsubsection{Section 14.3}\label{section-14.3}

\hypertarget{surveyElement143}{}

\hypertarget{surveyResult143}{}

Show Quiz Solution

\hypertarget{display.Quiz143.2}{}
\begin{center}\rule{0.5\linewidth}{\linethickness}\end{center}

\subsubsection{Section 14.5}\label{section-14.5}

\hypertarget{surveyElement145}{}

\hypertarget{surveyResult145}{}

Show Quiz Solution

\hypertarget{display.Quiz145.2}{}
\begin{center}\rule{0.5\linewidth}{\linethickness}\end{center}

\subsubsection{Section 14.6}\label{section-14.6}

\hypertarget{surveyElement146}{}

\hypertarget{surveyResult146}{}

Show Quiz Solution

\hypertarget{display.Quiz146.2}{}
\begin{center}\rule{0.5\linewidth}{\linethickness}\end{center}

\section{Exercises}\label{S:Exercises}

\subsection{The Issue}\label{the-issue}

Users expect to see exercises in a textbook. For \emph{Loss Data
Analytics}, we have many sources and types of exercises that we can draw
upon. The issue is deciding upon the type to re-inforce the emphasis of
the book.

Currently, the book features almost as many different exercise types as
there are authors. Some chapters have no exercises, others have an
extensive amount.

These differing viewpoints are also reflected in the examples in the
book. To illustrative, some chapters focus on mathmatical/statistical
developments, some on data analytic aspects, some on business content.
This is the good and bad associated with a heterogenous collection of
authors.

We can reduce the effects of this heterogeneity by providing a large
pool of exercises. Some may exist outside of the book. With a large pool
of exercises, users (teachers) can draw upon the subset that reflects
their area of interests.

\subsection{Wordpress Exercises}\label{wordpress-exercises}

Here is a series of exercises that guide the viewer through some of the
theoretical foundations of \emph{Loss Data Analytics}. Each tutorial is
based on one or more questions from the professional actuarial
examinations -- typically the Society of Actuaries short term actuarial
mathematics exam that used to be known as ``Exam C.'' These exercises
are coded using \textbf{Wordpress}.

We can explore bringing them into the \texttt{R} \textbf{Bookdown}
environment.

\subsubsection{2. Frequency Modeling}\label{frequency-modeling-1}

\href{https://www.ssc.wisc.edu/~jfrees/loss-data-analytics/loss-data-analytics-problems/}{Frequency
Distribution Guided Tutorials}

\subsubsection{3. Modeling Loss
Severity}\label{modeling-loss-severity-1}

\href{http://www.ssc.wisc.edu/~jfrees/loss-data-analytics/chapter-3-modeling-loss-severity/loss-data-analytics-severity-problems/}{Severity
Distribution Guided Tutorials}

\subsubsection{4. Model Selection and
Estimation}\label{model-selection-and-estimation-1}

\href{http://www.ssc.wisc.edu/~jfrees/loss-data-analytics/loss-data-analytics-model-selection/}{Model
Selection Guided Tutorials}

\subsubsection{5. Aggregate Loss Models}\label{aggregate-loss-models-1}

\href{https://www.ssc.wisc.edu/~jfrees/loss-data-analytics/aggregate-loss-guided-tutorials/}{Aggregate
Loss Guided Tutorials}

\section{Statistical Code}\label{S:StatisticalCode}

\begin{itemize}
\item
  As the baseline, we have already remarked that \texttt{R} statistical
  code appears in the text, although it does not overwhelm readers via
  the \emph{Show/Hide} features that we are using.
\item
  It will be straight forward to also include \textbf{Python} code in
  the same fashion.
\item
  We have a
  \href{https://ewfrees.github.io/LDARcode/index.html}{supporting site
  for statistical code} that allows viewers to get more experience
  analyzying loss data using \texttt{R}.
\item
  There are a variety of ways of learning coding procedures
  interactively. One option is
  \href{https://www.datacamp.com}{Datacamp}. In particular, they have
  developed an \texttt{R} package,
  \href{https://support.datacamp.com/hc/en-us/articles/360007749853-What-is-DataCamp-Light-}{Datacamp
  Light}. This allows us to integrate this interactive approach into our
  authoring system.
  \href{https://ewfreesres.github.io/RegressModel/index.html}{Here} is a
  link to an online regression tutorial done using this approach. There
  are a few little technical details but we think these are now ironed
  out. This is a promising option that will help make the book special.
\end{itemize}


\end{document}
